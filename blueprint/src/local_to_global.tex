\chapter{From local to global}%
\label{cha:from_local_to_global}

In this chapter, we gather some topological preliminaries allowing to build
global objects from local ones. This is usually not discussed in informal expositions
where such arguments are either implicit or mixed with more specific arguments.

We first need to discuss how to build a function having everywhere some local
properties from a sequence of functions having those properties on bigger and
bigger parts of the source space. We actually want to also accomodate finite sequences
so we start with a definition of the source of our sequences.

\begin{definition}
  \label{def:index_type}\lean{index_type}\leanok
  For every natural number $N$ we set
  \[
    \IT{N} =
    \begin{cases}
      ℕ \text{ if $N = 0$}\\
      \{0, \dots, N - 1\} \text{otherwise}
    \end{cases}
  \]
\end{definition}

On each $\IT{N}$ we use the obvious linear ordering. In particular there is no
maximal elementwhen $N = 0$ and $N-1$ is maximal if $N$ is positive.
The successor function $S \co \IT{N} → \IT{N}$ is the function sending $n$ to
$n+1$ unless $n$ is maximal, in which case $S(n) = n$.

Our first lemma gives a criterion ensuring that a sequence of functions is locally
ultimately constant hence has a limit that locally ultimately agrees with the
elements of the sequence.
Remember that a family of sets $V_i$ in a topological space $X$ is locally finite if
every point of $X$ has a neighborhood that intersects only finitely many $V_i$.

\begin{lemma}
  \label{lem:exists_forall_eventually_of_index_type}
  \lean{locally_finite.exists_forall_eventually_of_index_type}
  \leanok
  Let $X$ be a topological space and let $Y$ be any set. Let
  $f$ be a sequence of functions from $X$ to $Y$ indexed by $\IT{N}$
  for some $N$. Let $V$ be a family of subsets of $X$ indexed by $\IT{N}$
  such that for every non-maximal $n$, $f_{S(n)}$ coincides with $f_n$ outside
  $V_{S(n)}$. If $V$ is locally finite then there exists $F \co X → Y$ such
  that, for every $x$ and every sufficiently large $n$,
  $F$ coincides with $f_n$ near $x$.
\end{lemma}

\begin{proof}\leanok
  The assumption that $V$ is locally finite gives, for every $x$ in $X$, a
  subset $U_x$ of $X$ such that $U_x$ is a neighborhood of $x$ and intersects
  only finite many $V_n$'s. In particular we can find an upper bound $n₀(x)$ of the set
  of indices $i$ in $\IT{N}$ such that $V_n$ intersects $U_X$.
  Since, for every non-maximal $n$, $f_{S(n)}$ coincides with $f_n$ outside
  $V_{S(n)}$, we get by induction that, for all $n ≥ n₀(x)$, $f_n$ coincides
  with $f_{n₀(x)}$ on $U_x$.

  We now define $F$ as $x ↦ f_{n₀(x)}(x)$. We claim that, for every $x$,
  $F$ coincides with $f_n$ on $U_x$ as soon as $n$ is at least $n₀(x)$.
  Indeed let us fix $x$ and $n ≥ n₀(x)$ and $y ∈ U_x$. We have
  \begin{align*}
    f_n(y) &= f_{n₀(x)}(y) \text{ since $n ≥ n₀(x)$ and $y ∈ U_x$}\\
           &= f_{\max(n₀(x), n₀(y))}(y) \text{ since $\max(n₀(x), n₀(y)) ≥ n₀(x)$ and $y ∈ U_x$}\\
           &= f_{n₀(y)}(y) \text{ since $\max(n₀(x), n₀(y)) ≥ n₀(y)$ and $y ∈ U_y$}
           &= F(y) \text{ by definition of $F$}.
  \end{align*}
\end{proof}

In the preceding lemma, the limit function $F$ inherits all local properties of
the elements of the sequence. In order to make this precise, we need the
language of germs of functions. One can defines germs with respects to any filter but
we will need only the case of neighborhood filters~: two functions $f$ and $g$
define the same germ at some point $x$ if they coincide near $x$.

\begin{definition}\label{def:germ}\lean{filter.germ}\leanok
  Let $X$ be a topological space, $x$ a point in $X$ and $Y$ a set. A germ of function
  from $X$ to $Y$ at $x$ is an element of the quotient $(X → Y)_x$ of the set of functions
  from $X$ to $Y$ by the relation $f ∼ g$ if $f$ and $g$ coincide near $x$. The image
  of a function $f$ in this quotient will be denoted by $[f]_x$.

  A local predicate on functions from $X$ to $Y$ is a family of predicates on the
  germ set $(X → Y)_x$ for every $x$ in $X$.
\end{definition}

For instance if $Y$ is also equipped with a topology then continuity is
(equivalent to) a local predicate on functions from $X$ to $Y$ since a function
is continuous if and only if it is continuous at every point $x$ and this
condition only depends on the germ of the function at $x$.

\begin{lemma}
  \uses{def:germ}\lean{inductive_construction}\leanok
  Let $X$ be a topological space and $Y$ be any set. Let $U$ be a locally
  finite family of subsets of $X$ indexed by some $\IT{N}$. Let $P₀$ be a local
  predicate on functions from $X$ to $Y$, let $i ↦ P₁^i$ be a family of such
  predicates, and let $i ↦ P₂^i$ be a family of predicates on functions from $X$ to
  $Y$, both families being indexed by $\IT{N}$. Assume there exists $f₀ \co X → Y$
  such $P₀ [f₀]_x$ holds for every $x$ and $P₂^0 f₀$ holds. Assume that for
  every $i$ in $\IT{N}$ and every $f \co X → Y$ such that $P₂^i f$ and
  $P₀ [f]_x$ and $P₁^j [f]_x$ hold for every $x$ and for every $j < i$, there
  exists a function $f' \xo X → Y$ such that  $P₀ [f]_x$ and $P₁^j [f]_x$ hold
  for every $x$ and for every $j ≤ i$ and $P₂^{S(i)} f$ holds unless $i$ is maximal and
  $f'$ coincides with $f$ outside $U_i$.
  Then there exists $f \co X → Y$ which everywhere satisfies $P₀$ and all $P₁^i$'s.
\end{lemma}

\begin{proof}
  \leanok\uses{lem:exists_forall_eventually_of_index_type}
  The main assumption from the lemma allows to build by induction a sequence
  $f$ of functions from $X$ to $Y$ indexed by $\IT{N}$ such that, for every $n ∈ \IT{N}$,
  \begin{itemize}
    \item $f_n$ satisfies $P₀$ everywhere
    \item for every $i ≤ n$, $f_n$ satisfies $P₁^i$ everywhere.
    \item $f_{S(n)}$ satisfies $P₂$ unless $n$ is maximal.
    \item $f_{S(n)}$ coincides with $f_n$ outside $U_{S(n)}$.
  \end{itemize}
  Note that the first term of this sequence isn't $f₀$ but the function obtained by
  applying the induction assumption to $f₀$.

  The preceding lemma applied to this sequence gives a map $f$ which locally
  coincides with every element which is far enough in the sequence.
  Let $x$ be a point in $X$. Let $n$ be large enough to ensure $f$ coincides with
  $f_n$ near $x$. By definition this means $[f]_x = [f_n]_x$ and we know $P₀ [f_n]_x$
  hence we get $P₀ [f]_x$. Now fix also $n$ in $\IT{N}$. Let $n'$ be large
  enough to be larger than $n$ and such that $[f]_x = [f_{n'}]_x$. Since $n' ≥ n$ we have
  $P₁^n [f_{n'}]_x$ hence $P₁^n [f]_x$.
\end{proof}
