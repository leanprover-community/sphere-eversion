\chapter{Loops}
\label{chap:loops}

\section{Introduction}
\label{sec:loops_introduction}

In this chapter, we explain how to construct families of loops to feed into the
corrugation process explained at the end of the introduction.
A loop is a map defined on the circle $𝕊^1 = ℝ/ℤ$ with values in a
finite-dimensional vector space.
It can also freely be seen as $1$-periodic maps defined on $ℝ$.

\begin{definition}
  \label{def:average}
  \lean{loop.average}
  \leanok
  The average of a loop $γ$ is $\bar γ := \int_{𝕊^1} γ(s)\, ds$.
\end{definition}

Throughout this document, $E$ and $F$ will denote finite-dimensional
real vector spaces.

\begin{definition}
  \label{def:loop_family_support}
  \lean{loop.support}\leanok
  The \emph{support} of a family $γ$ of loops in $F$ parametrized by $E$ is
  the closure of the set of $x$ in $E$ such that $γ_x$ is not a constant loop.
\end{definition}

All of this chapter is devoted to proving the following proposition.

\begin{proposition}
  \label{prop:loops}
  \lean{exists_loops}
  \leanok
  Let $K$ a compact set in $E$. Let $Ω$ be an open set in $E × F$ such that, for each $x$,
  $Ω_x := Ω ∩ (\{x\} × F)$ is connected in $\{x\} × F$.

  Let $β$ and $g$ be smooth maps from $E$ to $F$.
  Assume that $β(x) ∈ Ω_x$ for all $x$, and $g(x) = β(x)$ near $K$.

  If, for every $x$, $g(x)$ is in the convex hull of $Ω_x$, then there
  exists a smooth family of loops
  \[
    γ \co E × [0, 1] × 𝕊^1 → F, (x, t, s) ↦ γ^t_x(s)
  \]
  such that, for all $x ∈ E$, all $t ∈ ℝ$ and all  $s ∈ 𝕊^1$,
  \begin{itemize}
    \item
      $γ^t_x(s) ∈ Ω_x$
    \item
      $γ^0_x(s) = β(x)$
    \item
      $\bar γ^1_x = g(x)$
    \item
      $γ^t_x(s) = β(x)$ if $x$ is near $K$.
  \end{itemize}
\end{proposition}

Let us briefly sketch the geometric idea behind the above proposition
if we pretend there is only one point $x$, and drop it from the
notation, and also focus only on $γ^1$.
By assumption, there is a finite collection of points $p_i$ in $Ω$ and $λ_i ∈
[0, 1]$ such that $g$ is the barycenter $\sum λ_i p_i$. Since $Ω$ is open and
connected, there is a smooth loop $γ_0$ which goes through each $p_i$. The
claim is that $g$ is the average value of $γ = γ_0 ∘ h$ for some
self-diffeomorphism $h$ of $𝕊^1$. The idea is to choose $h$ such that
$γ$ rushes to $p_1$, stays there during a time roughly $λ_1$, rushes to
$p_2$, etc. But, in order to achieve average exactly $g$, it seems like $h$
needs to be a discontinuous piecewise constant map. The assumption that $Ω$ is
open means that the convex hull is open, which gives enough slack to get away with
a smooth $h$.

In the previous proof sketch, there is a lot of freedom in constructing $γ$,
which is problematic when trying to do it consistently when $x$ varies.

\section{Preliminaries}
\label{sec:preliminaries}

In this section, $E$ is a real vector space with (finite) dimension $d$.
We'll need the Carathéodory lemma:

\begin{lemma}[Carathéodory's lemma]
\label{lem:caratheodory}
\lean{convex_hull_eq_union}
\leanok
  If a point $x$ of $E$ lies in the convex hull of a set $P$, then $x$
  belongs to the convex hull of a finite set of affinely independent points
  of $P$.
\end{lemma}

\begin{proof}
  \leanok
  By assumption, there is a finite set of points $t_i$ in $P$ and
  weights $f_i$ such that $x = \sum f_i t_i$, each $f_i$ is non-negative
  and $\sum f_i = 1$.
  Choose such a set of points of minimum cardinality. We argue by
  contradiction that such a set must be affinely independent.

  Thus suppose that there is some vanishing combination $\sum g_i t_i$ with
  $\sum g_i = 0$ and not all $g_i$ vanish.
  Let $S = \{i | g_i > 0\}$.
  Let $i_0$ in $S$ be an index minimizing $f_i/g_i$. We shall obtain our
  contradiction by showing that $x$ belongs to the convex hull of the set
  $\{t_i| i \ne i_0\}$, which has cardinality strictly smaller than
  $\{t_i\}$.

  We thus define new weights $k_i = f_i - g_i f_{i_0}/g_{i_0}$.
  These weights sum to $\sum f_i - (\sum g_i)f_{i_0}/g_{i_0} = 1$
  and $k_{i_0} = 0$.
  Each $k_i$ is non-negative, thanks to the choice of $i_0$ if
  $i$ is in $S$ or using that $f_i$, $-g_i$ and $f_{i_0}/g_{i_0}$
  are all non-negative when $i$ is not in $S$.
  It remain to compute
  \begin{align*}
    \sum_{i ≠ i_0} k_i t_i &= \sum_i k_i t_i \\
      &= \sum_i (f_i - g_i f_{i_0}/g_{i_0}) t_i  \\
      &= \sum_i f_i t_i - \left(\sum_i g_i t_i\right)f_{i_0}/g_{i_0})   \\
      &= x
  \end{align*}
  where we use $k_{i_0} = 0$ in the first equality.
\end{proof}


\begin{definition}
  \label{def:surrounds_points}
  \lean{surrounding_pts}
  \leanok
  A point $x$ in $E$
  is surrounded by points $p_0$, \dots, $p_d$ if those points are
  affinely independent and there exist weights $w_i ∈ (0, 1)$ such
  that $x = \sum_i w_i p_i$.
\end{definition}

Note that, in the above definition, the number of points $p_i$ is fixed
by the dimension $d$ of $E$, and that the weights $w_i$ are the barycentric
coordinates of $x$ with respect to the affine basis $p_0, \ldots, p_d$.

\begin{lemma}
  \label{lem:interior_chab}
  \lean{interior_convex_hull_aff_basis}\leanok
  Given an affine basis $b$ of $E$, the interior of the convex hull of
  $b$ is the set of points with strictly positive barycentric coordinates.
\end{lemma}

\begin{proof}
  \leanok
  For each $i$, let:
  \[
    w_i \co E \to ℝ
  \]
  be the $i^{\rm th}$ barycentric coordinate with respect to the basis $b$.
  Since $E$ is finite-dimensional, each $w_i$ is a continuous open map. For
  such a map, the operation of taking interior commutes with preimage, and so
  we have:
  \begin{align*}
    \IntConv(b) &= \Int\left(\bigcap_i w_i^{-1}([0, \infty))\right)\\
                &= \bigcap_i \Int(w_i^{-1}([0, \infty))\\
                &= \bigcap_i w_i^{-1}(\Int([0, \infty))\\
                &= \bigcap_i w_i^{-1}((0, \infty))
  \end{align*}
  as required.
\end{proof}

\begin{lemma}
  \label{lem:int_homothety_cvx}
  \lean{convex.subset_interior_image_homothety_of_one_lt}\leanok
  Given a point $c$ of $E$ and a real number $t$, let:
  \[
    h^c_t \co E \to E
  \]
  be the homothety which dilates about $c$ by a scale of $t$.

  Suppose $c$ belongs to the interior of a convex subset $C$ of $E$
  and $t > 1$, then
  \[
    C \subseteq \Int(h^c_t(C))
  \]
\end{lemma}

\begin{proof}
  \leanok
  Since $h^c_t$ is a homeomorphism with inverse $h^c_{t^{-1}}$, taking $s = t^{-1}$,
  the required result is equivalent to showing:
  \[
    h^c_s(C) \subseteq \Int(C)
  \]
  where $s \in (0, 1)$.

  Let $x$ be a point of $C$, we must show there exists an open neighborhood $U$
  of $h^c_s(x)$, contained in $C$. In fact we claim:
  \[
    U = h^x_{1-s}(\Int(C))
  \]
  is such a set. Indeed $U$ is open since $h^x_{1-s}$ is a homeomorphism and $U$
  contains $h^c_s(x)$ since:
  \[
    h^c_s(x) = h^x_{1-s}(c) \in h^x_{1-s}(\Int(C))
  \]
  since $c$ belongs to $\Int(C)$. Finally:
  \[
    h^x_{1-s}(\Int(C)) \subseteq h^x_{1-s}(C) \subseteq C
  \]
  where the second inclusion follows since $C$ is convex and contains $x$.
\end{proof}

\begin{lemma}
  \label{lem:int_cvx}
  \lean{surrounded_of_convex_hull}\leanok
  If a point $x$ of $E$ lies in the convex hull of an open set $P$,
  then it is surrounded by some collection of points belonging to $P$.
\end{lemma}

\begin{proof}
  \leanok
  \uses{lem:caratheodory,lem:interior_chab,lem:int_homothety_cvx}
  It follows from \Cref{lem:interior_chab} that we need only show that $E$
  has an affine basis $b$ of points belonging to $P$ such that $x$ lies in
  the interior of the convex hull of $b$.

  Carathéodory's lemma~\ref{lem:caratheodory} provides affinely independent
  points $p_0, \dots, p_k$ in $P$ such that $x$ belongs to their convex
  hull. Since $P$ is open, we may extend $p_i$ to an affine basis
  \[
    \hat b = \{p_0, \ldots, p_d\},
  \]
  where all points still belong to $P$.
  Note that $x$ belongs to the convex hull of $\hat b$.

  Now let $c$ be a point in the interior of the convex hull of $\hat b$
  (e.g., the centroid) and for each $\epsilon > 0$, consider the homothety
  \[
    h_{1+\epsilon} \co E \to E,
  \]
  which dilates about $c$ by a scale of $1 + \epsilon$.

  Since $\hat b$ is finite and contained in $P$, and $P$ is open, there exists
  $\epsilon > 0$ such that
  \[
    h_{1+\epsilon} (\hat b) \subseteq P.
  \]
  We claim the required basis is:
  \[
    b = h_{1+\epsilon} (\hat b)
  \]
  for any such $\epsilon$. Indeed, applying
  \Cref{lem:int_homothety_cvx} to $\Conv(\hat b)$ we see:
  \begin{align*}
    x \in \Conv(\hat b) &\subseteq \Int (h_{1+\epsilon} (\Conv(\hat b)))\\
                  &= \Int (\Conv(h_{1+\epsilon} (\hat b)))
  \end{align*}
  as required.
\end{proof}


\begin{lemma}
  \label{lem:smooth_barycentric_coord}
  \uses{def:surrounds_points}
  \lean{smooth_surrounding}
  \leanok
  For every $x$ in $E$ and every collection of points $p ∈ E^{d+1}$
  surrounding $x$, there is a function $w \co E × E^{d+1} → ℝ^{d+1}$ such that,
  for every $(y, q)$ in a neighborhood of $(x, p)$,
  \begin{itemize}
    \item
      $w$ is smooth at $(y, q)$
    \item
      $w(y, q) > 0$
    \item
      $\sum_{i=0}^d w_i(y, q) = 1$
    \item
      $y = \sum_{i=0}^d w_i(y, q)q_i$
  \end{itemize}
\end{lemma}

\begin{proof}
  \leanok
  Let:
  \begin{align*}
    A = E \times \{ q \in E^{d+1} ~|~ \mbox{$q$ is an affine basis for $E$} \},
  \end{align*}
  and define:
  \begin{align*}
    w \co A &\to ℝ^{d+1}\\
    (y, q) &\mapsto \mbox{barycentric coordinates of $y$ with respect to $q$}.
  \end{align*}

  If we fix an affine basis $b$ of $E$, we may express $w$ as a
  ratio of determinants in terms of coordinates relative to $b$. More precisely,
  by Cramer's rule, if $0 \le i \le d$ and $w_i$ is the $i^{\rm th}$ component of $w$,
  then:
  \begin{align*}
    w_i (y, q) = \det M_i (y, q) / \det N (q)
  \end{align*}
  where $N(q)$ is the $(d+1)\times (d+1)$ matrix whose columns are the barycentric
  coordinates of the components of $q$ relative to $b$, and $M_i (y, q)$ is $N(q)$
  except with column $i$ replaced by the barycentric coordinates of $y$ relative
  to $b$.

  Since determinants are smooth functions and $(y, q) \mapsto \det N(q)$ is
  non-vanishing on $A$, $w$ is smooth on $A$.

  Finally define:
  \begin{align*}
    U = w^{-1}((0, \infty)^{d+1}),
  \end{align*}
  and note that $U$ is open in $A$, since it is the preimage of an open set under the
  continuous map $w$. In fact since $A$ is open, $U$ is open as a subset of $E \times E^{d+1}$.
  Note that $(x, p) \in U$ since $p$ surrounds $x$.

  We may extend $w$ to $E \times E^{d+1}$ by giving it any values at all outside $A$.
\end{proof}

\section{Constructing loops}

\subsection{Surrounding families}
\label{sub:surrounding_families}

It will be convenient to introduce some more vocabulary.

\begin{definition}
  \label{def:surrounds}
  \uses{def:surrounds_points}
  \lean{loop.surrounds}
  \leanok
  We say a loop $γ$ surrounds a vector $v$ if $v$ is surrounded
  by a collection of points belonging to the image of $γ$.
  Also, we fix a base point $0$ in $𝕊^1$ and say a loop is based at some
  point $b$ if $0$ is sent to $b$.
\end{definition}

The first main task in proving \Cref{prop:loops} is to construct
suitable families of loops $γ_x$ surrounding $g(x)$, by assembling local
families of loops.
Those will then be reparametrized to get the correct average in the next
section.
In this section, we will work only with \emph{continuous} loops.
This will make constructions easier and we will smooth those loops
in the end, taking advantage of the fact that $Ω$ and the surrounding
condition are open.

Thanks to Carathéodory's lemma, constructing \emph{one} such loop
with values in some open $O$ is easy as soon as $v$ belongs to the
convex hull of $O$.

\begin{lemma}
  \label{lem:loop_of_hull}
  \uses{def:surrounds}
  \lean{surrounding_loop_of_convex_hull}
  \leanok
  If a vector $v$ is in the convex hull of a connected open subset $O$
  then, for every base point $b ∈ O$, there is a continuous
  family of loops
  $γ \co [0, 1] × 𝕊^1 → E, (t, s) ↦ γ^t(s)$ such that, for all $t$ and
  $s$:
  \begin{itemize}
    \item
      $γ^t$ is based at $b$
    \item
      $γ^0(s) = b$
    \item
      $γ^t(s) ∈ O$
    \item
      $γ^1$ surrounds $v$
  \end{itemize}
\end{lemma}

\begin{proof}
  \uses{lem:int_cvx}
  \leanok
  Since $O$ is open, \Cref{lem:int_cvx} gives points $p_i$ in $O$
  surrounding $x$.
  Since $O$ is open and connected, it is path connected.
  Let $λ \co [0, 1] → Ω_x$ be a continuous path starting at $b$ and
  going through the points $p_i$.
  We can concatenate $λ$ and its opposite to get $γ^1$,
  say $γ^1(s) = λ((1-\cos 2πs)/2)$.
  This is a round-trip loop: it back-tracks when it reaches $λ(1)$
  at $s = 1/2$.
  We then define $γ^t$ as the round-trip that stops at $s = t/2$, stays
  still until $s = 1-t/2$ and then backtracks.
\end{proof}


\begin{definition}
  \label{def:family_surrounds}
  \uses{def:surrounds}
  \lean{surrounding_family}
  \leanok
  A continuous family of loops $γ \co E × [0, 1] × 𝕊^1 → F, (x, t, s) ↦ γ^t_x(s)$
  surrounds a map $g \co E → F$ with base $β \co E → F$
  on $U ⊂ E$ in $Ω ⊂ E × F$ if, for every $x$ in $U$, every $t ∈ [0, 1]$ and every
  $s ∈ 𝕊^1$,
  \begin{itemize}
    \item
      $γ^t_x$ is based at $β(x)$
    \item
      $γ^0_x(s) = β(x)$
    \item
      $γ^1_x$ surrounds $g(x)$
    \item
      $(x,γ^t_x(s)) ∈ Ω$.
  \end{itemize}
  The space of such families will be denoted by
  $\Loop(g, β, U, Ω)$.
\end{definition}

Families of surrounding loops are easy to construct locally.

\begin{lemma}
  \label{lem:local_loops}
  \uses{def:family_surrounds}
  \lean{local_loops}
  \leanok
  Assume $Ω$ is open and connected over some neighborhood of $x_0$.
  If $g(x)$ is in the convex hull of $Ω_x$ for $x$ near $x_0$ then there is
  a continuous family of loops defined near $x_0$, based at $β$,
  taking value in $Ω$ and surrounding $g$.
\end{lemma}

\begin{proof}
  \leanok
  \uses{lem:loop_of_hull, lem:smooth_barycentric_coord}
  In this proof we don't mention the $t$ parameter since it plays no
  role, but it is still there.
  \Cref{lem:loop_of_hull} gives a loop $γ$ based at $β(x_0)$,
  taking values in $Ω_{x_0}$ and surrounding $g(x_0)$.
  We set $γ_x(s) = β(x) + (γ(s) - β(x_0))$.
  Each $γ_x$ takes values in $Ω_x$ because $Ω$ is open over some
  neighborhood of $x_0$.
  \Cref{lem:smooth_barycentric_coord} guarantees that this loop surrounds $g(x)$
  for $x$ close enough to $x_0$.
\end{proof}

The difficulty in constructing global families of surrounding loops is that
there are plenty of surrounding loops and we need to choose them
consistently.
The key feature of the above definition is that the $t$ parameter not only
allows us to cut out the corrugation
process in the next chapter, but also brings a ``satisfied or refund'' guarantee,
as explained in the next lemma.

\begin{lemma}
  \label{lem:satisfied_or_refund}
  \uses{def:family_surrounds}
  \lean{satisfied_or_refund}
  \leanok
  For every set $U ⊂ E$,  $\Loop(g, β, U, Ω)$ is path connected:
  for every $γ_0$ and $γ_1$ in $\Loop(g, β, U, Ω)$,
  there is a continuous map
  $δ \co [0, 1] × E × [0, 1] × 𝕊^1 → F, (τ, x, t, s) ↦ δ^t_{τ, x}(s)$
  which interpolates between $γ_0$ and $γ_1$ in
  $\Loop(g, β, U, Ω)$.
\end{lemma}


% I (Floris) renamed $γ_{τ, x}^t$ to $δ_{τ, x}^t$ so that we don't have a clash of notation where
% $γ_{0, x}^t(s)$ is defined to be $γ_{0, x}^{ρ(τ)t}left(\frac1{1 - τ} s\right)$

\begin{proof}
  \leanok
  Let $ρ$ be the piecewise affine map from $ℝ$ to $ℝ$ such that
  $ρ(τ) = 1$ if $τ ≤ 1/2$, $ρ$ is affine on $[1/2, 1]$,
  $ρ(τ) = 0$ if $τ ≥ 1$.
  We set
  \[
    δ_{τ, x}^t(s) =
    \begin{cases}
      γ_{0,x}^{ρ(τ)t}\left(\frac1{1 - τ} s\right) & \text{if $s ≤ 1 - τ$ and $τ < 1$}\\
      γ_{1,x}^{ρ(1-τ)t}\left(\frac1τ \big(s - (1- τ)\big)\right) &
      \text{if $s ≥ 1 - τ$ and $τ > 0$}\\
    \end{cases}
  \]
  It is clear that if $s = 1 - τ$ then both branches agree and are equal to $β(x)$.
  Therefore it is easy to see that $δ$ is continuous at $(τ, x, t, s)$
  except when $(τ,s)=(1,0)$ or $(τ,s)=(0,1)$.

  To show the continuity for $(τ,s)=(1,0)$, let $K$ be a compact neighborhood of $x$ in $E$.
  Then $γ_0$ is uniformly continuous on the compact set $K × [0, 1] × 𝕊^1$, which means that
  $γ_{0,x'}^t$ tends uniformly to the constant function $s ↦ β(x)$ as $(x', t)$ tends to
  $(x, 0)$.
  This means that $γ_{0,x'}^{ρ(τ)t'}$ tends uniformly to the constant function $s ↦ β(x)$
  as $(τ, x', t')$ tends to $(1, x, t)$. This means that $δ$ is continuous at $(τ,s)=(1,0)$
  (it is clear that the other branch also tends to $β(x)$). The continuity at $(τ,s)=(0,1)$ is
  entirely analogous.

  The beautiful observation motivating the above formula is why each
  $δ_{τ, x}^1$ surrounds $g(x)$.
  The key is that the image of $δ_{τ, x}^1$ contains the image of
  $γ_{0,x}^1$ when $τ ≤ 1/2$, and contains  the image of
  $γ_{1,x}^1$ when $τ ≥ 1/2$.
  Hence $δ_{τ, x}^1$ always surrounds $g(x)$.
\end{proof}

\begin{corollary}
  \label{cor:extend_loops}
  \uses{def:family_surrounds}
  \lean{extend_loops}
  \leanok
  Let $U_0$ and $U_1$ be open sets in $E$.
  Let $K_0 ⊂ U_0$ and $K_1 ⊂ U_1$ be compact subsets.
  For any $γ_0 ∈ \Loop(U_0, g, β, Ω)$ and $γ_1 ∈ \Loop(U_1, g, β, Ω)$,
  there exists $U ∈ 𝓝(K_0 ∪ K_1)$ and
  there exists $γ ∈ \Loop(U, g, β, Ω)$
  which coincides with $γ_0$ near $K_0\cup U_1^c$.
\end{corollary}

\begin{proof}
  \leanok
  \uses{lem:satisfied_or_refund}
  Let $U_0'$ be an open neighborhood of $K_0$ whose closure
  $\overline U_0'$ is compact in $U_0$.
  Since $\overline U_0'$ and $K_1' := K_1 ∖ (K_1 ∩ U_0)$
  are disjoint compact subsets of $E$, there is some continuous cut-off
  $ρ \co E → [0, 1]$ which vanishes on $U_0'$ and equals one on some
  neighborhood $U_1'$ of $K_1'$.

  \Cref{lem:satisfied_or_refund} gives a homotopy of loops
  $γ_τ$ from $γ_0$ to $γ_1$ on $U_0 ∩ U_1$. Moreover, note that $γ_τ$ is defined on all of $E$.
  On $U_0' ∪ (U_0 ∩ U_1) ∪ U_1'$, which is a
  neighborhood of $K_0 ∪ K_1$, we set
  \[
      γ_x = γ_{ρ(x), x}
  \]
  which has the required properties.
\end{proof}


\begin{lemma}
  \label{lem:∃_surrounding_loops}
  \uses{def:family_surrounds}
  \lean{exists_surrounding_loops}
  \leanok
  In the setup of \Cref{prop:loops}, assume we have a
  continuous family $γ$ of loops defined near $K$ which is based at $β$,
  surrounds $g$ and such that each $γ_x^t$ takes values in $Ω_x$.
  Then there such a family which is defined on all of $E$ and agrees
  with $γ$ near $K$.
\end{lemma}

\begin{proof}
  \uses{lem:local_loops, cor:extend_loops}
  \leanok
  Let $U_0$ be an open set containing $K$ such that $γ$ forms a surrounding family
  of loops on $U_0$.
  Let $U_i$, $i ≥ 1$ be a locally finite family of open sets with local
  surrounding families of loops $γ^i$ and compact subsets $K_i ⊂ U_i$ covering $E$.

  % \begin{explanation} % I am imagining that eventually this will be hidden until clicked
    This is possible, since by \Cref{lem:local_loops} there is a local surrounding family
    of loops around each point in $E$. Since $E$ is locally compact we may pick
    a compact neighborhood around each point in $E$ with such a local family of loops.
    Since $E$ is paracompact second countable, we can pick a countable refinement $U_i$
    which is locally finite and still covers $E$.
    By the shrinking lemma we can pick closed (and hence compact) subsets $K_i ⊂ U_i$ that
    also cover $E$.
  % \end{explanation}

  Now we define a family $(δ^i)_{i \in NN}$ by $δ^0 = \rst{γ}{U_0}$ and
  $δ^{i+1}$ is obtained by extending $δ^i$ using $γ^i$ via
  \Cref{cor:extend_loops}.
  Since $δ^{i+1}$ equals $δ^i$ on $U_i^c$, this sequence is locally eventually constant.
  Therefore it has a well-defined limit $δ$ which is defined on continuous on all $E$.
  Since $δ^{i+1}$ equals $δ^i$ on a neighborhood of $K\cup\bigcup_{j<i} K_i,$
  we know in particular that $δ^i$ equals $γ$ on a neighborhood of $K$.
  Since $K$ is compact, it has some neighborhood $O$
  such that $\rst{δ^i}{O}$ is eventually constant with eventual value $\rst{δ}{O}$.
  Hence $δ=γ$ on a neighborhood of $K$.
\end{proof}

\subsection{The reparametrization lemma}
\label{sub:the_reparametrization_lemma}

The second ingredient needed to prove \Cref{prop:loops} is a
parametric reparametrization lemma.

\begin{lemma}
\label{lem:reparametrization}
\uses{def:surrounds, def:average}
\lean{smooth_surrounding_family.reparametrize_average}
\leanok
Let $γ \co E × 𝕊^1 → F$ be a smooth family of loops surrounding
a map $g$.
There is a smooth family of equivalences $φ \co E × 𝕊^1 → 𝕊^1$ such
that each $γ_x ∘ φ_x$ has average $g(x)$ and $φ_x(0) = 0$.
\end{lemma}

\begin{proof}
  \uses{lem:smooth_barycentric_coord}
  \leanok
  For any fixed $x$, since $γ_x$ strictly surrounds $g(x)$, there are points
  $s_1, …, s_{n+1}$ in $𝕊^1$ such that $g(x)$ is surrounded
  by the corresponding points $γ_x(s_j)$.

  Let $μ_1, …, μ_{n+1}$ be smooth positive probability measures very
  close to the Dirac measures on $s_j$ (ie. $μ_j = f_j\, ds$ for some
  smooth positive function $f_j$ and, for any function $h$,
  $\int h\,dμ_j$ is almost $h(s_j)$).  We set $p_j = \int γ_x\, d\mu_j$, which
  is almost $γ_x(s_j)$ so that $g(x) = \sum w_j p_j$ for some weights
  $w_j$ in the open interval $(0, 1)$ according to
  \Cref{lem:smooth_barycentric_coord}.

  If $x'$ is in a sufficiently small neighborhood of $x$,
  \Cref{lem:smooth_barycentric_coord} gives smooth weight functions $w_j$
  such that $g(x') = \sum w_j(x')p_j(x')$.
  Let $U^i$, $i ≥ 1$ be a locally finite cover of $E$ by such
  neighborhoods, with corresponding measures $μ_j^i$, moving points
  $p_j^i$ and weight functions $w_j^i$.
  Let $(ρ_i)$ be a partition of unity associated to this covering. For
  every $x$, we set
  \[
    μ_x = \sum_{i=1}^∞ \sum_{j=1}^{n+1} ρ_i(x)w_j^i(x)\, μ_j^i
  \]
  so that:
  \begin{align*}
    \int γ_x\, dμ_x &=
    \sum_i ρ_i(x)\sum_{j=1}^{n+1} w_j^i(x) \int γ_x\, dμ_j^i\\
    &= \sum_i ρ_i(x)\sum_{j=1}^{n+1} w_j^i(x) p_j^i(x)\\
    &= \sum_i ρ_i(x) g(x) = g(x).
  \end{align*}
  We now set $φ_x^{-1}(t) = \int_0^tdμ_x$ so that
  $g(x) = \overline{γ_x ∘ φ_x}$ for all $x$.
\end{proof}

\subsection{Proof of the loop construction proposition}
\label{sub:proof_of_the_loop_construction_proposition}

We finally assemble the ingredients from the previous two sections.

\begin{proof}[Proof of \Cref{prop:loops}]
  \proves{prop:loops}
  \uses{lem:∃_surrounding_loops, lem:reparametrization}
  \leanok
  Let $γ^*$ be a family of loops surrounding the origin in $F$,
  constructed using \Cref{lem:local_loops}.
  For $x$ in some neighborhood $U^*$ of $K$ where $g = β$, we set
  $γ_x = g(x) + εγ^*$ where $ε > 0$ is sufficiently small to ensure
  the image of $γ_x$ and its convex hull are contained in $Ω_x$ (recall $Ω$ is
  open and $K$ is compact).
  \Cref{lem:∃_surrounding_loops} extends this family to a continuous
  family of surrounding loops $γ_x$ for all $x$ (this is not yet our
  final $γ$).

  We then need to approximate this continuous family by a smooth one.
  Some care is needed to ensure that it stays based at $β$.
  For instance, we can first compose each loop by some fixed surjective
  continuous map from $𝕊^1$ to itself that sends a neighborhood of $0$
  to $0$.
  This way each loop becomes constant near $0$, and a convolution
  smoothing will then keep the value at $0$.
  If the smoothing is sufficiently $C^0$ small then the new $γ$ is
  still surrounding and takes values in $Ω$.

  Then \Cref{lem:reparametrization} gives a family of circle
  diffeomorphisms $h_x$ such that $γ^1_x ∘ h_x$ has average $g(x)$.

  Finally we choose a cut-off function function $χ$ which vanishes near
  $K$ and equals one near $E ∖ U^*$.  In $U^*$, we replace
  $γ_x ∘ h_x = g(x) + γ^* ∘ h_x$ by $g(x) + χ(x)γ^* ∘ h_x$. This
  operation does not change the average values of these loops, because
  it rescales them around their average value, but makes them constant
  on $\Op{K}$. Also, those loops stay in $Ω$, thanks to our choice of $ε$.
\end{proof}

% vim: set expandtab sw=2 tabstop=2:
