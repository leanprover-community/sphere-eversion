\section{Loops}
\label{chap:loops}

\subsection{Introduction}
\label{sec:loops_introduction}

In this chapter, we explain how to construct families of loops to feed into the
corrugation process explained at the end of the introduction.

Throughout this document, $E$ and $F$ will denote finite-dimensional
real vector spaces.

\begin{definition}
  \label{def:loop}
  \lean{Loop, Loop.average, Loop.support}
  \leanok
  A loop is a map defined on the circle $𝕊^1 = ℝ/ℤ$ with values in a
  finite-dimensional vector space.
  It can also freely be seen as $1$-periodic maps defined on $ℝ$.

  The average of a loop $γ$ is $\bar γ := \int_{𝕊^1} γ(s)\, ds$.

  The \emph{support} of a family $γ$ of loops in $F$ parametrized by $E$ is
  the closure of the set of $x$ in $E$ such that $γ_x$ is not a constant loop.
\end{definition}

All of this chapter is devoted to proving the following proposition.

\begin{proposition}
  \label{prop:∃_loops}
  \lean{exist_loops}
  \uses{def:loop}
  \leanok
  Let $K$ a compact set in $E$. Let $Ω$ be an open set in $E × F$.

  Let $β$ and $g$ be smooth maps from $E$ to $F$.
  Write $Ω_x := \{ y ∈ F \mid (x, y) ∈ Ω\}$, assume that $β(x) ∈ Ω_x$ for all $x$,
  and that $g(x) = β(x)$ near $K$.

  If, for every $x$, $g(x)$ is in
  the convex hull of the connected component of $Ω_x$ containing $β(x)$,
  then there exists a smooth family of loops
  \[
    γ \co E × [0, 1] × 𝕊^1 → F, (x, t, s) ↦ γ^t_x(s)
  \]
  such that, for all $x ∈ E$, all $t ∈ ℝ$ and all  $s ∈ 𝕊^1$,
  \begin{itemize}
    \item
      $γ^t_x(s) ∈ Ω_x$
    \item
      $γ^0_x(s) = γ^t_x(1) = β(x)$
    \item
      $\bar γ^1_x = g(x)$
    \item
      $γ^t_x(s) = β(x)$ if $x$ is near $K$.
  \end{itemize}
\end{proposition}

Let us briefly sketch the geometric idea behind the above proposition
if we pretend there is only one point $x$, and drop it from the
notation, and also focus only on $γ^1$.
By assumption, there is a finite collection of points $p_i$ in $Ω$ and $λ_i ∈
[0, 1]$ such that $g$ is the barycenter $\sum λ_i p_i$. Since $Ω$ is open and
connected, there is a smooth loop $γ_0$ which goes through each $p_i$. The
claim is that $g$ is the average value of $γ = γ_0 ∘ h$ for some
self-diffeomorphism $h$ of $𝕊^1$. The idea is to choose $h$ such that
$γ$ rushes to $p_1$, stays there during a time roughly $λ_1$, rushes to
$p_2$, etc. But, in order to achieve average exactly $g$, it seems like $h$
needs to be a discontinuous piecewise constant map. The assumption that $Ω$ is
open means that the convex hull is open, which gives enough slack to get away with
a smooth $h$.

In the previous proof sketch, there is a lot of freedom in constructing $γ$,
which is problematic when trying to do it consistently when $x$ varies.

\subsection{Surrounding points}
\label{sec:preliminaries}

This section collects elementary results about convex sets in finite
dimensional real vector spaces that will help to construct families of loops.
In this section, $E$ is a real vector space with (finite) dimension $d$.
The discussion will center around the following definition which is tailored to
our ulterior needs.

\begin{definition}
  \label{def:surrounds_points}
  \lean{SurroundingPts}
  \leanok
  A point $x$ in $E$
  is surrounded by points $p_0$, \dots, $p_d$ if those points are
  affinely independent and there exist weights $w_i ∈ (0, 1)$ with sum $1$
  such that $x = \sum_i w_i p_i$.
\end{definition}

Note that, in the above definition, the number of points $p_i$ is fixed
by the dimension $d$ of $E$, and that the weights $w_i$ are the barycentric
coordinates of $x$ with respect to the affine basis $p_0, \ldots, p_d$.

The first important point in this definition is that surrounding is
smoothly locally stable: if $x$ is surrounded by a collection of points $p$
then points that are close to $y$ are surrounded by every collection of points
$q$ that is closed to $p$, and the relevant barycentric coordinates smoothly
depend on $y$ and $q$. The precise statement follows.

\begin{lemma}
  \label{lem:smooth_barycentric_coord}
  \uses{def:surrounds_points}
  \lean{smooth_surrounding}
  \leanok
  For every $x$ in $E$ and every collection of points $p ∈ E^{d+1}$
  surrounding $x$, there is a function $w \co E × E^{d+1} → ℝ^{d+1}$ such that,
  for every $(y, q)$ in a neighborhood of $(x, p)$,
  \begin{itemize}
    \item
      $w$ is smooth at $(y, q)$
    \item
      $w(y, q) > 0$
    \item
      $\sum_{i=0}^d w_i(y, q) = 1$
    \item
      $y = \sum_{i=0}^d w_i(y, q)q_i$.
  \end{itemize}
\end{lemma}

\begin{proof}
  \leanok
  Let:
  \begin{align*}
    A = E \times \{ q \in E^{d+1} ~|~ \mbox{$q$ is an affine basis for $E$} \},
  \end{align*}
  and define:
  \begin{align*}
    w \co A &\to ℝ^{d+1}\\
    (y, q) &\mapsto \mbox{barycentric coordinates of $y$ with respect to $q$}.
  \end{align*}

  If we fix an affine basis $b$ of $E$, we may express $w$ as a
  ratio of determinants in terms of coordinates relative to $b$. More precisely,
  by Cramer's rule, if $0 \le i \le d$ and $w_i$ is the $i^{\rm th}$ component of $w$,
  then:
  \begin{align*}
    w_i (y, q) = \det M_i (y, q) / \det N (q)
  \end{align*}
  where $N(q)$ is the $(d+1)\times (d+1)$ matrix whose columns are the barycentric
  coordinates of the components of $q$ relative to $b$, and $M_i (y, q)$ is $N(q)$
  except with column $i$ replaced by the barycentric coordinates of $y$ relative
  to $b$.

  Since determinants are smooth functions and $(y, q) \mapsto \det N(q)$ is
  non-vanishing on $A$, $w$ is smooth on $A$.

  Finally define:
  \begin{align*}
    U = w^{-1}((0, \infty)^{d+1}),
  \end{align*}
  and note that $U$ is open in $A$, since it is the preimage of an open set under the
  continuous map $w$. In fact since $A$ is open, $U$ is open as a subset of $E \times E^{d+1}$.
  Note that $(x, p) \in U$ since $p$ surrounds $x$.

  We may extend $w$ to $E \times E^{d+1}$ by giving it any values at all outside $A$.
\end{proof}

Then we need a criterion ensuring a point $x$ is surrounded by a collection of points taken
in a given subset $P$. The first temptation is to hope that $x$ being in the interior of
the convex hull of $P$ is enough. But this is not true. For instance the center of a square
in a plane is in the interior of the convex hull of the set $P$ of vertices of the square,
but it isn't surrounded by any set of vertices. This counter example also shows that
the stability lemma above is slightly less trivial than it sounds.

The rest of this section is devoted to the following result that proves no such issue arises
when $P$ is open.

\begin{proposition}
  \label{prop:surrounded_by_open}
  \uses{def:surrounds_points}
  \lean{surrounded_of_convexHull}\leanok
  If a point $x$ of $E$ lies in the convex hull of an open set $P$,
  then it is surrounded by some collection of points belonging to $P$.
\end{proposition}

This proposition will be proven at the end of this section.
We'll first need the Carathéodory lemma:

\begin{lemma}[Carathéodory's lemma]
\label{lem:caratheodory}
\lean{convexHull_eq_union}
\leanok
  If a point $x$ of $E$ lies in the convex hull of a set $P$, then $x$
  belongs to the convex hull of a finite set of affinely independent points
  of $P$.
\end{lemma}

\begin{proof}
  \leanok
  By assumption, there is a finite set of points $t_i$ in $P$ and
  weights $f_i$ such that $x = \sum f_i t_i$, each $f_i$ is non-negative
  and $\sum f_i = 1$.
  Choose such a set of points of minimum cardinality. We argue by
  contradiction that such a set must be affinely independent.

  Thus suppose that there is some vanishing combination $\sum g_i t_i$ with
  $\sum g_i = 0$ and not all $g_i$ vanish.
  Let $S = \{i | g_i > 0\}$.
  Let $i_0$ in $S$ be an index minimizing $f_i/g_i$. We shall obtain our
  contradiction by showing that $x$ belongs to the convex hull of the set
  $\{t_i| i \ne i_0\}$, which has cardinality strictly smaller than
  $\{t_i\}$.

  We thus define new weights $k_i = f_i - g_i f_{i_0}/g_{i_0}$.
  These weights sum to $\sum f_i - (\sum g_i)f_{i_0}/g_{i_0} = 1$
  and $k_{i_0} = 0$.
  Each $k_i$ is non-negative, thanks to the choice of $i_0$ if
  $i$ is in $S$ or using that $f_i$, $-g_i$ and $f_{i_0}/g_{i_0}$
  are all non-negative when $i$ is not in $S$.
  It remain to compute
  \begin{align*}
    \sum_{i ≠ i_0} k_i t_i &= \sum_i k_i t_i \\
      &= \sum_i (f_i - g_i f_{i_0}/g_{i_0}) t_i  \\
      &= \sum_i f_i t_i - \left(\sum_i g_i t_i\right)f_{i_0}/g_{i_0})   \\
      &= x
  \end{align*}
  where we use $k_{i_0} = 0$ in the first equality.
\end{proof}



\begin{lemma}
  \label{lem:interior_chab}
  \lean{AffineBasis.interior_convexHull}\leanok
  Given an affine basis $b$ of $E$, the interior of the convex hull of
  $b$ is the set of points with strictly positive barycentric coordinates.
\end{lemma}

\begin{proof}
  \leanok
  For each $i$, let:
  \[
    w_i \co E \to ℝ
  \]
  be the $i^{\rm th}$ barycentric coordinate with respect to the basis $b$.
  Since $E$ is finite-dimensional, each $w_i$ is a continuous open map. For
  such a map, the operation of taking interior commutes with preimage, and so
  we have:
  \begin{align*}
    \IntConv(b) &= \Int\left(\bigcap_i w_i^{-1}([0, \infty))\right)\\
                &= \bigcap_i \Int(w_i^{-1}([0, \infty))\\
                &= \bigcap_i w_i^{-1}(\Int([0, \infty))\\
                &= \bigcap_i w_i^{-1}((0, \infty))
  \end{align*}
  as required.
\end{proof}

\begin{lemma}
  \label{lem:int_homothety_cvx}
  \lean{Convex.subset_interior_image_homothety_of_one_lt}\leanok
  Given a point $c$ of $E$ and a real number $t$, let:
  \[
    h^c_t \co E \to E
  \]
  be the homothety which dilates about $c$ by a scale of $t$.

  Suppose $c$ belongs to the interior of a convex subset $C$ of $E$
  and $t > 1$, then
  \[
    C \subseteq \Int(h^c_t(C))
  \]
\end{lemma}

\begin{proof}
  \leanok
  Since $h^c_t$ is a homeomorphism with inverse $h^c_{t^{-1}}$, taking $s = t^{-1}$,
  the required result is equivalent to showing:
  \[
    h^c_s(C) \subseteq \Int(C)
  \]
  where $s \in (0, 1)$.

  Let $x$ be a point of $C$, we must show there exists an open neighborhood $U$
  of $h^c_s(x)$, contained in $C$. In fact we claim:
  \[
    U = h^x_{1-s}(\Int(C))
  \]
  is such a set. Indeed $U$ is open since $h^x_{1-s}$ is a homeomorphism and $U$
  contains $h^c_s(x)$ since:
  \[
    h^c_s(x) = h^x_{1-s}(c) \in h^x_{1-s}(\Int(C))
  \]
  since $c$ belongs to $\Int(C)$. Finally:
  \[
    h^x_{1-s}(\Int(C)) \subseteq h^x_{1-s}(C) \subseteq C
  \]
  where the second inclusion follows since $C$ is convex and contains $x$.
\end{proof}

We are now ready to come back to \Cref{prop:surrounded_by_open}.

\begin{proof}[Proof of \Cref{prop:surrounded_by_open}]\proves{prop:surrounded_by_open}
  \leanok
  \uses{lem:caratheodory,lem:interior_chab,lem:int_homothety_cvx}
  It follows from \Cref{lem:interior_chab} that we need only show that $E$
  has an affine basis $b$ of points belonging to $P$ such that $x$ lies in
  the interior of the convex hull of $b$.

  Carathéodory's lemma~\ref{lem:caratheodory} provides affinely independent
  points $p_0, \dots, p_k$ in $P$ such that $x$ belongs to their convex
  hull. Since $P$ is open, we may extend $p_i$ to an affine basis
  \[
    \hat b = \{p_0, \ldots, p_d\},
  \]
  where all points still belong to $P$.
  Note that $x$ belongs to the convex hull of $\hat b$.

  Now let $c$ be a point in the interior of the convex hull of $\hat b$
  (e.g., the centroid) and for each $\epsilon > 0$, consider the homothety
  \[
    h_{1+\epsilon} \co E \to E,
  \]
  which dilates about $c$ by a scale of $1 + \epsilon$.

  Since $\hat b$ is finite and contained in $P$, and $P$ is open, there exists
  $\epsilon > 0$ such that
  \[
    h_{1+\epsilon} (\hat b) \subseteq P.
  \]
  We claim the required basis is:
  \[
    b = h_{1+\epsilon} (\hat b)
  \]
  for any such $\epsilon$. Indeed, applying
  \Cref{lem:int_homothety_cvx} to $\Conv(\hat b)$ we see:
  \begin{align*}
    x \in \Conv(\hat b) &\subseteq \Int (h_{1+\epsilon} (\Conv(\hat b)))\\
                  &= \Int (\Conv(h_{1+\epsilon} (\hat b)))
  \end{align*}
  as required.
\end{proof}


\subsection{Constructing loops}

\subsubsection*{Surrounding families}

It will be convenient to introduce some more vocabulary.

\begin{definition}
  \label{def:surrounds}
  \uses{def:loop, def:surrounds_points}
  \lean{Loop.Surrounds}
  \leanok
  We say a loop $γ$ surrounds a vector $v$ if $v$ is surrounded
  by a collection of points belonging to the image of $γ$.
  Also, we fix a base point $0$ in $𝕊^1$ and say a loop is based at some
  point $b$ if $0$ is sent to $b$.
\end{definition}

The first main task in proving \Cref{prop:∃_loops} is to construct
suitable families of loops $γ_x$ surrounding $g(x)$, by assembling local
families of loops.
Those will then be reparametrized to get the correct average in the next
section.
In this section, we will work only with \emph{continuous} loops.
This will make constructions easier and we will smooth those loops
in the end, taking advantage of the fact that $Ω$ and the surrounding
condition are open.

Thanks to Carathéodory's lemma, constructing \emph{one} such loop
with values in some open $O$ is easy as soon as $v$ belongs to the
convex hull of $O$.

\begin{lemma}
  \label{lem:loop_of_hull}
  \uses{def:surrounds}
  \lean{surrounding_loop_of_convexHull}
  \leanok
  If a vector $v$ is in the convex hull of a connected open subset $O$
  then, for every base point $b ∈ O$, there is a continuous
  family of loops
  $γ \co [0, 1] × 𝕊^1 → E, (t, s) ↦ γ^t(s)$ such that, for all $t$ and
  $s$:
  \begin{itemize}
    \item
      $γ^t$ is based at $b$
    \item
      $γ^0(s) = b$
    \item
      $γ^t(s) ∈ O$
    \item
      $γ^1$ surrounds $v$
  \end{itemize}
\end{lemma}

\begin{proof}
  \uses{prop:surrounded_by_open}
  \leanok
  Since $O$ is open, \Cref{prop:surrounded_by_open} gives points $p_i$ in $O$
  surrounding $x$.
  Since $O$ is open and connected, it is path connected.
  Let $λ \co [0, 1] → Ω_x$ be a continuous path starting at $b$ and
  going through the points $p_i$.
  We can concatenate $λ$ and its opposite to get $γ^1$.
  This is a round-trip loop: it back-tracks when it reaches $λ(1)$
  at $s = 1/2$.
  We then define $γ^t$ as the round-trip that stops at $s = t/2$, stays
  still until $s = 1-t/2$ and then backtracks.
\end{proof}


\begin{definition}
  \label{def:family_surrounds}
  \uses{def:surrounds}
  \lean{SurroundingFamily}
  \leanok
  A continuous family of loops $γ \co E × [0, 1] × 𝕊^1 → F, (x, t, s) ↦ γ^t_x(s)$
  surrounds a map $g \co E → F$ with base $β \co E → F$
  on $U ⊂ E$ in $Ω ⊂ E × F$ if, for every $x$ in $U$, every $t ∈ [0, 1]$ and every
  $s ∈ 𝕊^1$,
  \begin{itemize}
    \item
      $γ^t_x$ is based at $β(x)$
    \item
      $γ^0_x(s) = β(x)$
    \item
      $γ^1_x$ surrounds $g(x)$
    \item
      $(x,γ^t_x(s)) ∈ Ω$.
  \end{itemize}
  The space of such families will be denoted by
  $\Loop(g, β, U, Ω)$.
\end{definition}

Families of surrounding loops are easy to construct locally.

\begin{lemma}
  \label{lem:local_loops}
  \uses{def:family_surrounds}
  \lean{local_loops}
  \leanok
  Assume $Ω$ is open over some neighborhood of $x_0$.
  If $g(x_0)$ is in the convex hull of the connected component of $Ω_{x_0}$ containing $β(x_0)$,
  then there is a continuous family of loops defined near $x_0$, based at $β$,
  taking value in $Ω$ and surrounding $g$.
\end{lemma}

\begin{proof}
  \leanok
  \uses{lem:loop_of_hull, lem:smooth_barycentric_coord}
  In this proof we don't mention the $t$ parameter since it plays no
  role, but it is still there.
  \Cref{lem:loop_of_hull} gives a loop $γ$ based at $β(x_0)$,
  taking values in $Ω_{x_0}$ and surrounding $g(x_0)$.
  We set $γ_x(s) = β(x) + (γ(s) - β(x_0))$.
  Each $γ_x$ takes values in $Ω_x$ because $Ω$ is open over some
  neighborhood of $x_0$.
  \Cref{lem:smooth_barycentric_coord} guarantees that this loop surrounds $g(x)$
  for $x$ close enough to $x_0$.
\end{proof}

The difficulty in constructing global families of surrounding loops is that
there are plenty of surrounding loops and we need to choose them
consistently.
The key feature of the above definition is that the $t$ parameter not only
allows us to cut out the corrugation
process in the next chapter, but also brings a ``satisfied or refund'' guarantee,
as explained in the next lemma.

\begin{lemma}
  \label{lem:satisfied_or_refund}
  \uses{def:family_surrounds}
  \lean{satisfied_or_refund}
  \leanok
  For every set $U ⊂ E$,  $\Loop(g, β, U, Ω)$ is ``path connected'':
  for every $γ_0$ and $γ_1$ in $\Loop(g, β, U, Ω)$,
  there is a continuous map
  $δ \co [0, 1] × E × [0, 1] × 𝕊^1 → F, (τ, x, t, s) ↦ δ^t_{τ, x}(s)$
  which interpolates between $γ_0$ and $γ_1$ in
  $\Loop(g, β, U, Ω)$.
\end{lemma}

The construction below morally proves that each $\Loop(g, β, U, Ω)$ is contractible, but
we will not even specify a topology on those spaces.
The definition of ``path connected'' in quotation marks is the above specific statement,
and only this statement will be used.

\begin{proof}
  \leanok
  Let $ρ$ be the piecewise affine map from $ℝ$ to $ℝ$ such that
  $ρ(τ) = 1$ if $τ ≤ 1/2$, $ρ$ is affine on $[1/2, 1]$,
  $ρ(τ) = 0$ if $τ ≥ 1$.
  We set
  \[
    δ_{τ, x}^t(s) =
    \begin{cases}
      γ_{0,x}^{ρ(τ)t}\left(\frac1{1 - τ} s\right) & \text{if $s ≤ 1 - τ$ and $τ < 1$}\\
      γ_{1,x}^{ρ(1-τ)t}\left(\frac1τ \big(s - (1- τ)\big)\right) &
      \text{if $s ≥ 1 - τ$ and $τ > 0$}\\
    \end{cases}
  \]
  It is clear that if $s = 1 - τ$ then both branches agree and are equal to $β(x)$.
  Therefore it is easy to see that $δ$ is continuous at $(τ, x, t, s)$
  except when $(τ,s)=(1,0)$ or $(τ,s)=(0,1)$.

  To show the continuity for $(τ,s)=(1,0)$, let $K$ be a compact neighborhood of $x$ in $E$.
  Then $γ_0$ is uniformly continuous on the compact set $K × [0, 1] × 𝕊^1$, which means that
  $γ_{0,x'}^t$ tends uniformly to the constant function $s ↦ β(x)$ as $(x', t)$ tends to
  $(x, 0)$.
  This means that $γ_{0,x'}^{ρ(τ)t'}$ tends uniformly to the constant function $s ↦ β(x)$
  as $(τ, x', t')$ tends to $(1, x, t)$. This means that $δ$ is continuous at $(τ,s)=(1,0)$
  (it is clear that the other branch also tends to $β(x)$). The continuity at $(τ,s)=(0,1)$ is
  entirely analogous.

  The beautiful observation motivating the above formula is why each
  $δ_{τ, x}^1$ surrounds $g(x)$.
  The key is that the image of $δ_{τ, x}^1$ contains the image of
  $γ_{0,x}^1$ when $τ ≤ 1/2$, and contains  the image of
  $γ_{1,x}^1$ when $τ ≥ 1/2$.
  Hence $δ_{τ, x}^1$ always surrounds $g(x)$.
\end{proof}

\begin{corollary}
  \label{cor:extend_loops}
  % \uses{def:family_surrounds} % causes transitive arrow in depgraph
  \lean{extend_loops}
  \leanok
  Let $U_0$ and $U_1$ be open sets in $E$.
  Let $K_0 ⊂ U_0$ and $K_1 ⊂ U_1$ be compact subsets.
  For any $γ_0 ∈ \Loop(U_0, g, β, Ω)$ and $γ_1 ∈ \Loop(U_1, g, β, Ω)$,
  there exists a neighborhood $U$ of $K_0 ∪ K_1$ and
  there exists $γ ∈ \Loop(U, g, β, Ω)$
  which coincides with $γ_0$ near $K_0\cup U_1^c$.
\end{corollary}

\begin{proof}
  \leanok
  \uses{lem:satisfied_or_refund}
  Let $C_0 = K_0\cup U_1^c$ and $C_1 := K_1 ∖ U_0$. Since $C_0$ and $C_1$ are disjoint closed sets,
  there is some continuous cut-off $ρ \co E → [0, 1]$
  which vanishes on a neighborhood of $C_0$ and equals one on a neighborhood of $C_1$.

  \Cref{lem:satisfied_or_refund} gives a homotopy of loops
  $γ_τ$ from $γ_0$ to $γ_1$ on $U_0 ∩ U_1$. Moreover, note that $γ_τ$ is defined on all of $E$.
  On $U_0' ∪ (U_0 ∩ U_1) ∪ U_1'$, which is a
  neighborhood of $K_0 ∪ K_1$, we set
  \[
      γ_x = γ_{ρ(x), x}
  \]
  which has the required properties.
\end{proof}

\begin{lemma}
  \label{lem:∃_surrounding_loops}
  % \uses{def:family_surrounds} % causes transitive arrow in depgraph
  \lean{exists_surrounding_loops}
  \leanok
  In the setup of \Cref{prop:∃_loops}, assume we have a
  continuous family $γ$ of loops defined near $K$ which is based at $β$,
  surrounds $g$ and such that each $γ_x^t$ takes values in $Ω_x$.
  Then there such a family which is defined on all of $E$ and agrees
  with $γ$ near $K$.
\end{lemma}

\begin{proof}
  \uses{lem:local_loops, cor:extend_loops, lem:relative_inductive_construction_of_loc}
  \leanok
  \Cref{lem:local_loops} proves the existence of local families of surrounding
  loops and \Cref{cor:extend_loops} allows to patch such families hence
  \Cref{lem:relative_inductive_construction_of_loc} proves global existence.
\end{proof}

\subsubsection*{The reparametrization lemma}

The second ingredient needed to prove \Cref{prop:∃_loops} is a
parametric reparametrization lemma. Gromov's original proof of this lemma makes
explicit use of a partition of unity. Motivated in particular by formalization
purposes, we will first state more abstract versions whose statements do not involve
any partition of unity but directly state a local-to-global property.


\begin{lemma}
\label{lem:reparametrization}
\uses{def:surrounds}
\lean{SmoothSurroundingFamily.reparametrize_average}
\leanok
Let $γ \co E × 𝕊^1 → F$ be a smooth family of loops surrounding
a map $g$.
There is a smooth family $φ \co E × 𝕊^1 → 𝕊^1$ such
that each $γ_x ∘ φ_x$ has average $g(x)$ and $φ_x(0) = 0$.
\end{lemma}

\begin{proof}
  \uses{lem:smooth_barycentric_coord, lem:exists_cont_diff_of_convex₂}
  \leanok
  Gromov's main idea in order to prove this result is to translate the problem
  of constructing a family of circle maps $φ$ into the problem of constructing
  a family of smooth density functions $f$ on the circle. We introduce some
  vocabulary in order to describe this reduction.
  Let $f \co E × ℝ → ℝ$ be a smooth positive function that is $1$-periodic in
  its second argument.
  We say that $f$ is a centering density for $(γ, g)$ at $x$ if
  $f_x \co ℝ → ℝ$ has average value one when seen as a function on $𝕊¹$ and
  the average value of $f_x γ_x$ is $g(x)$. We claim that, in order to prove
  the lemma, it is sufficient to build such an $f$ which is centering at every
  $x$. Indeed, assume we have such an $f$. We then get a smooth family
  of $ℤ$-equivariant functions $ψ \co E × ℝ → ℝ$ defined by
  $ψ_x(t) = \int_0^tf_x(s)ds$. Because $ψ$ is smooth and each $ψ_x$ is strictly
  monotone and $ℤ$-equivariant, one can check there is a smooth map
  $φ : E × ℝ → ℝ$ which is $ℤ$-equivariant and such that $φ_x ∘ ψ_x = \Id$ for
  each $x$. Seen as a family of functions from $𝕊¹$ to $𝕊¹$, those functions
  are suitable since, for every $x$, the change of variable formula gives:
  \[
    \int_{𝕊¹} γ_x ∘ φ_x(s)ds = \int_{𝕊¹} ψ_x'(s) γ_x ∘ φ_x(ψ_x(s))ds
     = \int_{𝕊¹} f_x(s) γ_x(s)ds = g(x).
  \]

  We now prove the existence of a function which is a centering density at every point of $x$.
  For any given $x$, this constraint is clearly convex.
  Hence \Cref{lem:exists_cont_diff_of_convex₂} ensures it is enough to prove existence
  of functions that are centering densities in a neighborhood of any given point $x$.
  So we fix some $x$ in $E$.

  Since $γ_x$ strictly surrounds $g(x)$, there are points
  $s_1, …, s_{n+1}$ in $𝕊^1$ such that $g(x)$ is surrounded
  by the corresponding points $γ_x(s_j)$.


  Let $f_1, …, f_{n+1}$ be smooth positive periodic maps from $ℝ$ to $ℝ$
  which average value $1$ on a period and such that the corresponding measures
  on $𝕊¹$ are very
  close to the Dirac measures on $s_j$, ie. for any function $h$, the average value of
  $f_jh$ is almost $h(s_j)$. We set $p_j = \int f_jγ_x\, ds$, which
  is almost $γ_x(s_j)$ so that $g(x) = \sum w_j p_j$ for some weights
  $w_j$ in the open interval $(0, 1)$ according to
  \Cref{lem:smooth_barycentric_coord}.

  If $x'$ is in a sufficiently small neighborhood of $x$,
  \Cref{lem:smooth_barycentric_coord} gives smooth weight functions $w_j$
  such that $g(x') = \sum w_j(x')p_j(x')$. Hence we can set
  $f_{x'}(s) = \sum w_j(x')f_j(s)$.
\end{proof}

\subsubsection*{Proof of the loop construction proposition}

We finally assemble the ingredients from the previous two sections.

\begin{proof}[Proof of \Cref{prop:∃_loops}]
  \proves{prop:∃_loops}
  \uses{lem:∃_surrounding_loops, lem:reparametrization, lem:exists_cont_diff_of_convex}
  \leanok
  Let $γ^*$ be a family of loops surrounding the origin in $B_F(0,1)$
  the open unit ball in $F$, constructed using \Cref{lem:local_loops}.
  For $x$ in some neighborhood $U^*$ of $K$ where $g = β$, we set
  $γ_x = g(x) + εγ^*$ where $ε > 0$ is sufficiently small to ensure that
  $B_{E\times F}((x,β(x)),2ε)\subseteq \Omega$ (recall $Ω$ is
  open and $K$ is compact).
  \Cref{lem:∃_surrounding_loops} extends this family to a continuous
  family of surrounding loops $γ_x$ for all $x$ (this is not yet our
  final $γ$).

  We then need to approximate this continuous family by a smooth one.
  Some care is needed to ensure that it stays based at $β$.
  We can first reparametrize $γ$ on $[0,1] \times 𝕊^1$
  to ensure that $γ$ is constant in a neighborhood of
  $C = \{(t, s) \in [0,1] \times 𝕊^1 \mid t = 0 \text{ or } s = 0\}$.
  Using \Cref{lem:exists_cont_diff_of_convex}, we can find a smooth function
  that has distance at most $ε$ from $γ$ and coincides with $γ$ on $C$
  (using the fact that $γ$ is already smooth near $C$).
  Since all loops that are sufficiently close to $γ$ still surround $g$,
  we can also ensure that the new smoothened $γ$ is still surrounding.

  Then \Cref{lem:reparametrization} gives a family of circle
  diffeomorphisms $h_x$ such that $γ^1_x ∘ h_x$ has average $g(x)$.

  Finally we choose a cut-off function function $χ$ which vanishes near
  $E ∖ U^*$ and equals one near $K$. As our final family of loops,
  we choose $χ(x)g(x) + (1-χ(x))(γ_x ∘ h_x)$. This
  operation does not change the average values of these loops, because
  it rescales them around their average value, but makes them constant
  near $K$. Also, those loops stay in $Ω$, thanks to our choice of $ε$.
\end{proof}

% vim: set expandtab sw=2 tabstop=2:
