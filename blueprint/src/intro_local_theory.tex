\section{Main definitions and result}

The goal of this chapter is to explain the local aspects of
(Theillière's implementation of) convex integration, the next chapter will
cover global aspects. However convex integration can be completely hidden as a
proof detail. In this section we will give all definitions and statements that
will be used in the next chapter, and convex integration does not appear.
It is useful to know that nothing else is required later, but of course the
assumptions of the main result will be rather mysterious to readers who
completely skip the next sections of this chapter.

\begin{definition}
  \label{def:rel_loc}
  \leanok
  \lean{RelLoc, RelLoc.FormalSol}
  A first order differential relation for maps from $E$ to $F$ is a
  subset $\Rel$ of $E × F × \Hom(E, F)$.

  A solution of $\Rel$ is a function $f : E → F$ such that, for every $x$ in
  $E$, $(x, f(x), Df(x))$ is in $\Rel$.

  A \emph{formal} solution of $\Rel$ is a map
  $\F = (f, φ) \co E → F × \Hom(E, F)$ such that, for every $x$,
  $(x, f(x), φ(x))$ is in $\Rel$.
\end{definition}

Until the end of this chapter, $\Rel$ will always denote a first order
differential relation for maps from $E$ to $F$.

The first component of a map $\F \co E → F × \Hom(E, F)$ will sometimes
be denoted by $\bs \F \co E → F$ and called the base map of $\F$.

Note that every solution $f$ of a $\Rel$ leads to the formal solution $(f, Df)$.
The following definition distinguishes functions from $E$ to $F × \Hom(E, F)$
that have this shape.

\begin{definition}
  \label{def:hol_loc}
  \leanok
  \lean{JetSec.IsHolonomicAt}
  Let $E$ and $F$ be real normed spaces.
  A map $\F = (f, φ) : E → F × \Hom(E, F)$ is holonomic if,
  for every $x$, $Df(x) = φ(x)$.
\end{definition}

With these definitions we can state that our goal is to deform formal solutions
into holonomic ones in order to get solutions. The main result of this chapter
says that some criterion on the geometry of $\Rel$ ensures such deformations
exist. The following notion enters the description of this geometric criterion.

\begin{definition}
  \label{def:ample_subset}
  \leanok
  \lean{AmpleSet}
  A subset $Ω$ of a real vector space $E$ is ample if the convex
  hull of each connected component of $Ω$ is the whole $E$.
\end{definition}

\begin{definition}
  \label{def:ample_relation_loc}
  \leanok
  \lean{RelLoc.IsAmple}
  \uses{def:ample_subset, def:rel_slice}
  A first order differential relation $\Rel$ is ample if, for every $(x, y, φ)$
  in $E × F × \Hom(E, F)$ and for every hyperplane $H$ in $E$, the set
  \[
    \Big\{ ψ ∈ \Hom(E, F) \mid 
      ψ_{|H} = φ_{|H} \text{ and } (x, y, φ) ∈ \Rel\Big\}
  \]
  is ample.
\end{definition}

We can now state precisely how this above geometric criterion allows to locally
deform formal solutions into holonomic ones. In this statement we have nested
compact sets in $E$ and a formal solution that is already holonomic near some
closed set. We want to make it holonomic near the small compact set while
keeping it near the closed set and outside the large compact set.

\begin{proposition}
  \label{lem:h_principle_open_ample_loc}
  \leanok
  \lean{RelLoc.FormalSol.improve}
  \uses{def:ample_relation_loc, def:holonomic_loc}
  Let $\F$ be a formal solution of $\Rel$.
  Let $K_1 ⊂ E$ be a compact subset, and let $K_0$ be a compact subset of
  the interior of $K_1$.
  Assume $\F$ is holonomic near a closed subset $C$ of $E$.
  Let $ε$ be a positive real number.

  If $\Rel$ is open and ample then there is a homotopy $\F_t$ such that:
  \begin{enumerate}
    \item
      $\F_0 = \F$
    \item
      $\F_t$ is a formal solution of $\Rel$ for all $t$~;
    \item
      $\F_t(x) = \F(x)$ for all $t$ when $x$ is near $C$ or outside $K_1$.
    \item
      $d(\bs\F_t(x), \bs \F(x)) ≤ ε$ for all $t$ and all $x$~;
    \item
      $\F_1$ is holonomic near $K_0$~;
    \item
      $t ↦ F_t$ is constant near $0$ and $1$.
  \end{enumerate}
\end{proposition}

The above result is really the heart of the $h$-principle for open and ample
first order differential relations.
It will be used in the next chapter to get the global result for maps between
manifolds.
But we will also see that it is enough to deduce Smale’s sphere eversion
theorem, which is indeed about compact sets in vector spaces so require nothing
more global.
